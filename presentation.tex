%%%%%%%%%%%%%%%%%%%%%%%%%%%%%%%%%%%%%%%%%%%%%%%%%%%%%%%%%%%%%%%%%%%%%%%
%% ihf Beamer-Vorlage
%%     IHF, TUHH
%%
%% Author: Christian Rave, Thomas Jaschke, Christian Friesicke
%% Version: presentation.tex 2016.03
%% 			ihfBeamer.cls    2016.02
%%          ihf.sty          2016.02
%% Änderungen:
%% 		22.10.2012: CR 2012.01, ihfBeamer.cls -> Daten aus dem TUHH-Beamer-Template übernommen
%%		20.12.2012: CR 2012.02, ihfBeamer.cls -> Optionen "article" und "book"
%%      25.07.2013: CR 2013.01, ihfBeamer.cls -> Optionen "classic", "conference" und "student"
%%								Dazu die  Befehle "\iType", "\iPlace" und "\iSession"
%%		04.02.2014:CR 			Kodierung bei allen Dateien zu ISO-8859-1 geändert
%%      03.03.2014:CR 2014.01   ihfBeamer.cls -> Option "hideallsubsections" hinzugefügt
%%		27.06.2014:CR 2014.02   Umstellung auf utf8
%%								Optionen utf8 (default),latin1,anisnew,applemac
%%		24.11.2015:CR 2015.01   Auflistung eingebundener Packages vollständig
%%								Nicht benötigte entfernt
%%		02.11.2015:CR 2015.03   \includesvg-Befehl in ihf.sty ergänzt, Beispielfolien angepasst
%%		05.02.2016:CR 2016.01   Beispielfolien ergänzt, Option openFullScreen, Befehle \iPDFpagemode und \iPDFstartview ergänzt, Warnungen behoben
%%      04.03.2016:CR 2016.02   TUHH in Technische Universität Hamburg (ohne Harburg) geändert
%%      17.05.2016:TJ 2016.03   Änderung in ihf.sty
%% 		04.10.2017:DS 2017.01   Option "16zu9" in ihfBeamer.cls
%% 		04.10.2017:DS 2018.08   svg ausgelagert und um "external" Funktion erweitert
%%%%%%%%%%%%%%%%%%%%%%%%%%%%%%%%%%%%%%%%%%%%%%%%%%%%%%%%%%%%%%%%%%%%%%%

%\documentclass[]{ihfBeamer} %Präsentation (klassischer Stil)
\documentclass[student]{ihfBeamer} %Präsentation im ihf-Stil
%\documentclass[student,16zu9]{ihfBeamer} %Präsentation im 16:9-ihf-Stil
%\documentclass[conference]{ihfBeamer} %Präsentation im  Konferenz-Stil
%\documentclass[handout,student]{ihfBeamer} %Handout
%\documentclass[article,english]{ihfBeamer} %Artikel (article od book)

% Optionen für die ihfBeamer-Klasse:
%		classic,student,conference: Stil der Titelfolie
%		16zu9: Präsentation im 16:9-Format
%		handout: Für Ausdrucke (keine Farbigen Ränder etc.)
%		handout2: 2 Folien pro Seite
%					Bei Einblendungen (z.B. mit \only<2>) werden alle Zustände auf einmal gezeigt.
%								Hier hilft \only<2| handout:0>
%		article,book: Für Umdrucke
%		english,ngerman,german: Sprache
%		hideallsubsections: Keine Subsection auf Contentframe
%		utf8 (default),latin1,anisnew,applemac: Zeichencodieung. Wenn nicht utf8 verwendet wird, sind auf der Titelseite keine Umlaute zulässig (auch nicht für den Autornamen)
%		openFullScreen: PDF wird im Vollbildmodus geöffnet (nicht bei handout, article oder book, hier kann der Befehl \iPDFpagemode{FullScreen} verwendet werden)

% Anm.: Die ihfBeamer-Klasse bindet diese Packages ein:
%			graphicx    - Für Grafiken
%			amsmath     - Mathematischer Formelsatz
%			babel       - Spracheinstellung
%			inputenc    - Zeichencodierung
%			fontenc(T1) - Zeichencodierung	
%			lmodern     - Zeichensatz

%%%%%%%%%%%
%% Themes
\usetheme[numbers, compress]{Harburg}
\useinnertheme{circles}


%%%%%%%%%%
%% Pakete einbinden, je nach Bedarf
									
\usepackage{ihf}		
%	Die Option [pdfwarningpagegroup] schaltet die "PDF inclusion: multiple pdfs with page group included in a single page"-Warnung ein
%	Bindet diese Pakete ein:
%			upgreek   - Aufrechte griechische Buchstaben
%			blindtext - Blindtext
%			siunitx   - (SI)-Einheiten
%			import    - Wird für .svg-Bilder benötigt
%			acronym   - Abkürzungsverzeichnisse
%			adjustbox - Skalieren von Objekten
%			etoolbox  - Erweitere Befehle


\usepackage[]{svg} %Für svg-Bilder
%\usepackage[external]{svg} %  PDFs von allen Grafiken in den Ordner \ext erstellen

%	Verwendung von svg-Grafiken:
%		Voraussetzungen:
%			-Der Pfad zu Inkscape muss in der Umgebungsvariable path hinterlegt sein
%			-pdfLaTeX mit der Option "-shell-escape" ausgeführt werden
%		Befehl:
%		-> \includesvg[scale]{dir/}{filename}{ids}
%		Parameter:
%			scale, optional: Skalierung, wie bei \includegraphics
%			dir: Pfad (immer mit "/" nach dem letzten Ordnernamen)
%			filename: Dateiname (immer ohne ".svg")
%			ids: IDs der svg-Layer/Objekte.
%		Ohne ids:
%			\includesvg[scale]{dir/}{filename}{}
%			Zeichnet alle Layer des svg
%		Mit ids:
%			\includesvg[scale]{dir/}{filename}{layer1,layer2,layer3}
%			Zeichnet nur layer1, layer2 und layer3. Alle Layer werden in der gegebenen Reihenfolge gezeichnet.
%		Bsp:
%			\includesvg[width=0.5\textwidth]{pic/Bilder/}{Test}{layer2, layer3}
%			Zeichnet layer2 und layer3
%	Anm: Ein alternativ kann auch \svg verwendet werden


%%
%%%%%%%%%%

%%%%%%%%%%%%%%
%% Einstellungen für das Dokument
\iTitle[Kurz]{Langer Titel}
\iSubtitle[K. U.Titel]{Untertitel}
\iAuthor[D.Autor]{Der Author}
\iDate{\today}
%\iSeminar % == \iType{Seminarvortrag}
%\iType{Seminarvortrag} % Oder \iBachelor etc, siehe ihf-Template für Abschlussarbeiten

%% Beispiel für Konferenzen
%\iAuthor[D.Autor, D.Düsentrieb]{Der Author$^1$, Daniel Düsentrieb$^2$}
%\iInstitute{$^1$Technische Universität Hamburg\\
%$^2$Technische Universität Entenhausen}
%\iType{EuMW2012}
%\iPlace{Amsterdam, The Netherlands}
%\iSession{EuMIC08}

%% Mauelles Einstellen der Logos auf der Titelseite (z.B. für Konferenzen)
%\titlegraphic{\includegraphics[width=0.2\paperwidth]{./beamerthemeHarburg_tuhh_logo}
%\hspace{0.5cm}
%\includegraphics[width=0.2\paperwidth]{./ihf_logo}}


%%%%%%%%%%%%%%
%% Einstellungen für Listen:
\setbeamertemplate{itemize item}{$\bullet$} 
\setbeamertemplate{itemize subitem}{$\circ$}
\setbeamertemplate{itemize subsubitem}{$\blacktriangleright$}

%%%%%%%%%%%%%%
%% Document
\begin{document}


%%%%%%%%%%%%%%%%%%%%%%%%%%%%%%%%%%%%%%%%%%%%%%%%%%%%%%%%%%%%%%%%%%%%%%%%%%%%%%%%
% Titlepage
\titleframe
% Titlepage
%%%%%%%%%%%%%%%%%%%%%%%%%%%%%%%%%%%%%%%%%%%%%%%%%%%%%%%%%%%%%%%%%%%%%%%%%%%%%%%%

%%%%%%%%%%%%%%%%%%%%%%%%%%%%%%%%%%%%%%%%%%%%%%%%%%%%%%%%%%%%%%%%%%%%%%%%%%%%%%%%
% Table of Contents
\contentframe
% Table of Contents
%%%%%%%%%%%%%%%%%%%%%%%%%%%%%%%%%%%%%%%%%%%%%%%%%%%%%%%%%%%%%%%%%%%%%%%%%%%%%%%%

\section{Die ihf-Beamer Klasse}
\contentframe

\subsection{Optionen}

\begin{frame}{Übersicht}
Optionen der Klasse:\\
\begin{itemize}
\item Stile der Titelseite:
\begin{itemize}
\item classic: Nur TUHH-Logo
\item student: IHF-Logo, Art des Vortrages
\item conference: TUHH-Logo, Konferenz, Ort, Session
\end{itemize}
\item Foliensatz
\begin{itemize}
\item handout, handout2: Handout
\item article, book: Für Artikel, Skripte, Bücher
\end{itemize}
\item Sprache: english, ngerman, german
\item Zeichenkodierung: utf8 (default), latin1, anisnew, applemac
\item hideallsubsections: Subsections werden in der Gliederung nicht gezeigt
\item openFullScreen: PDF wird im Vollbildmodus geöffnet (nicht bei Handouts oder Umdrucken)
\end{itemize}
\end{frame}



\section{Svg-Grafiken}
\contentframe

\subsection{Befehle}

\begin{frame}{Voraussetzungen}
Die svg-Befehle starten Inkscape, um svg-Grafiken in pdf-Dateien umzuwandeln, dafür müssen folgende Bedingungen erfüllt sein:\\
\begin{itemize}
\item Inkscape muss installiert sein
\item Das System muss den Dateipfad zu Inksape kennen (Umgebungsvariable \textit{path})
\item (pdf)\LaTeX{} muss die Option \textit{-shell-escape} übergeben werden
\end{itemize}
\end{frame}

\begin{frame}{Übersicht}
Das ihf-Package stellt die folgenden Befehle zum Einbinden von svg-Grafiken bereit:\\
\begin{itemize}
\item \textbackslash includesvg [scale]\{dir/\}\{filename\}\{ids\}: Universalbefehl, neu ab v2015.3
\item \textbackslash svg [scale]\{dir/\}\{filename\}\{ids\}: Derselbe Befehl, nur kürzer
\item Veraltet:
\begin{itemize}
\item \textbackslash subincludesvg\{pic/\}\{Test\}
\item \textbackslash subincludesvgId\{pic/\}\{Test\}\{Id\}\{postfix\}
\item \textbackslash subincludesvgLayer\{--select=layer1 --select=layer3\}\{pic/\}\{Test\}\{postfix\}
\end{itemize}
\end{itemize}
\end{frame}

\subsection{Einfache Grafiken}

\begin{frame}{\textbackslash subincludesvg}
\subincludesvg{pic/}{Test}\\
\textbackslash subincludesvg\{pic/\}\{Test\}\\
Zeigt alle Ebenen auf einmal
\end{frame}

\begin{frame}{\textbackslash includesvg ohne ids}
\includesvg{pic/}{Test}{}\\
\textbackslash includesvg\{pic/\}\{Test\}\{\}\\
Macht dasselbe
\end{frame}

\begin{frame}{\textbackslash includesvg mit Skalierung}
\includesvg[width=0.5\textwidth]{pic/}{Test}{}\\
\textbackslash includesvg[width=0.5\textbackslash textwidth]\{pic/\}\{Test\}\{\}\\
Skalierung wie mit \textbackslash includegraphics möglich
\end{frame}

\subsection{Verwendung von Layern}

\begin{frame}{\textbackslash includesvg mit ids}
\begin{columns}[b]
\begin{column}{7cm}
\only<1| handout:0>{\svg{pic/}{Test}{layer1,layer2}}%
\only<2| handout:0|article:0>{\svg{pic/}{Test}{layer1,layer2,layer3}}%
\only<3| handout:0>{\svg{pic/}{Test}{layer1,layer2,layer3,layer4}}%
\only<4| handout:0|article:0>{\svg{pic/}{Test}{layer1,layer2,layer4}}%
\only<5>{\svg{pic/}{Test}{layer1,layer4,layer2}}%
\only<6>{\svg{pic/}{Test}{layer1,layer4,layer1:over}}%
\end{column}
\begin{column}{4cm}
Ebenen Schrittweise einblenden.

(handout:0: Nicht auf das Handout)

\begin{itemize}
\item<1,2,3> Layer aufdecken
\item<4> Und kombinieren
\item<5> Es kommt auf die Reihenfolge an.
\item<6> Mit : auf Dateien mit Suffix zugreifen.
\end{itemize}
\end{column}
\end{columns}
\only<1| handout:0>{\textbackslash includesvg\{pic/\}\{Test\}\{layer1,layer2\}}%
\only<2| handout:0|article:0>{\textbackslash includesvg\{pic/\}\{Test\}\{layer1,layer2,layer3\}}%
\only<3| handout:0>{\textbackslash includesvg\{pic/\}\{Test\}\{layer1,layer2,layer3,layer4\}}%
\only<4| handout:0|article:0>{\textbackslash includesvg\{pic/\}\{Test\}\{layer1,layer2,layer4\}}%
\only<5>{\textbackslash includesvg\{pic/\}\{Test\}\{layer1,layer4,layer2\}}%
\only<6>{\textbackslash includesvg\{pic/\}\{Test\}\{layer1,layer4,layer1:over\}}%
\end{frame}

\section{Weiteres}
\contentframe
\subsection{Blöcke}

\begin{frame}{Blöcke}
\begin{block}{Ein Block}
\alert<2>{Inhalt.}\\
\uncover<3|article:0>{Irgendwas}
\end{block}

\begin{exampleblock}{Ein Beispielblock}
Inhalt.
\end{exampleblock}

\begin{alertblock}{Ein Alarmblock}
Inhalt.
\end{alertblock}

\end{frame}



\end{document}

