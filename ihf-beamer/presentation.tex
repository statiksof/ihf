%%%%%%%%%%%%%%%%%%%%%%%%%%%%%%%%%%%%%%%%%%%%%%%%%%%%%%%%%%%%%%%%%%%%%%%
%% ihf Beamer-Vorlage
%%     IHF, TUHH
%%
%% Author: Christian Rave, Thomas Jaschke, Christian Friesicke
%% Version: presentation.tex 2016.03
%% 			ihfBeamer.cls    2016.02
%%          ihf.sty          2016.02
%% Änderungen:
%% 		22.10.2012: CR 2012.01, ihfBeamer.cls -> Daten aus dem TUHH-Beamer-Template übernommen
%%		20.12.2012: CR 2012.02, ihfBeamer.cls -> Optionen "article" und "book"
%%      25.07.2013: CR 2013.01, ihfBeamer.cls -> Optionen "classic", "conference" und "student"
%%								Dazu die  Befehle "\iType", "\iPlace" und "\iSession"
%%		04.02.2014:CR 			Kodierung bei allen Dateien zu ISO-8859-1 geändert
%%      03.03.2014:CR 2014.01   ihfBeamer.cls -> Option "hideallsubsections" hinzugefügt
%%		27.06.2014:CR 2014.02   Umstellung auf utf8
%%								Optionen utf8 (default),latin1,anisnew,applemac
%%		24.11.2015:CR 2015.01   Auflistung eingebundener Packages vollständig
%%								Nicht benötigte entfernt
%%		02.11.2015:CR 2015.03   \includesvg-Befehl in ihf.sty ergänzt, Beispielfolien angepasst
%%		05.02.2016:CR 2016.01   Beispielfolien ergänzt, Option openFullScreen, Befehle \iPDFpagemode und \iPDFstartview ergänzt, Warnungen behoben
%%      04.03.2016:CR 2016.02   TUHH in Technische Universität Hamburg (ohne Harburg) geändert
%%      17.05.2016:TJ 2016.03   Änderung in ihf.sty
%% 		04.10.2017:DS 2017.01   Option "16zu9" in ihfBeamer.cls
%% 		04.10.2017:DS 2018.08   svg ausgelagert und um "external" Funktion erweitert
%%%%%%%%%%%%%%%%%%%%%%%%%%%%%%%%%%%%%%%%%%%%%%%%%%%%%%%%%%%%%%%%%%%%%%%

%\documentclass[]{ihfBeamer} %Präsentation (klassischer Stil)
\documentclass[student,english]{ihfBeamer} %Präsentation im ihf-Stil
%\documentclass[student,16zu9]{ihfBeamer} %Präsentation im 16:9-ihf-Stil
%\documentclass[conference]{ihfBeamer} %Präsentation im  Konferenz-Stil
%\documentclass[handout,student]{ihfBeamer} %Handout
%\documentclass[article,english]{ihfBeamer} %Artikel (article od book)

% Optionen für die ihfBeamer-Klasse:
%		classic,student,conference: Stil der Titelfolie
%		16zu9: Präsentation im 16:9-Format
%		handout: Für Ausdrucke (keine Farbigen Ränder etc.)
%		handout2: 2 Folien pro Seite
%					Bei Einblendungen (z.B. mit \only<2>) werden alle Zustände auf einmal gezeigt.
%								Hier hilft \only<2| handout:0>
%		article,book: Für Umdrucke
%		english,ngerman,german: Sprache
%		hideallsubsections: Keine Subsection auf Contentframe
%		utf8 (default),latin1,anisnew,applemac: Zeichencodieung. Wenn nicht utf8 verwendet wird, sind auf der Titelseite keine Umlaute zulässig (auch nicht für den Autornamen)
%		openFullScreen: PDF wird im Vollbildmodus geöffnet (nicht bei handout, article oder book, hier kann der Befehl \iPDFpagemode{FullScreen} verwendet werden)

% Anm.: Die ihfBeamer-Klasse bindet diese Packages ein:
%			graphicx    - Für Grafiken
%			amsmath     - Mathematischer Formelsatz
%			babel       - Spracheinstellung
%			inputenc    - Zeichencodierung
%			fontenc(T1) - Zeichencodierung	
%			lmodern     - Zeichensatz

%%%%%%%%%%%
%% Themes
\usetheme[numbers, compress]{Harburg}
\useinnertheme{circles}


%%%%%%%%%%
%% Pakete einbinden, je nach Bedarf
									
\usepackage{ihf}		
%	Die Option [pdfwarningpagegroup] schaltet die "PDF inclusion: multiple pdfs with page group included in a single page"-Warnung ein
%	Bindet diese Pakete ein:
%			upgreek   - Aufrechte griechische Buchstaben
%			blindtext - Blindtext
%			siunitx   - (SI)-Einheiten
%			import    - Wird für .svg-Bilder benötigt
%			acronym   - Abkürzungsverzeichnisse
%			adjustbox - Skalieren von Objekten
%			etoolbox  - Erweitere Befehle


\usepackage[]{svg} %Für svg-Bilder
%\usepackage[external]{svg} %  PDFs von allen Grafiken in den Ordner \ext erstellen

%	Verwendung von svg-Grafiken:
%		Voraussetzungen:
%			-Der Pfad zu Inkscape muss in der Umgebungsvariable path hinterlegt sein
%			-pdfLaTeX mit der Option "-shell-escape" ausgeführt werden
%		Befehl:
%		-> \includesvg[scale]{dir/}{filename}{ids}
%		Parameter:
%			scale, optional: Skalierung, wie bei \includegraphics
%			dir: Pfad (immer mit "/" nach dem letzten Ordnernamen)
%			filename: Dateiname (immer ohne ".svg")
%			ids: IDs der svg-Layer/Objekte.
%		Ohne ids:
%			\includesvg[scale]{dir/}{filename}{}
%			Zeichnet alle Layer des svg
%		Mit ids:
%			\includesvg[scale]{dir/}{filename}{layer1,layer2,layer3}
%			Zeichnet nur layer1, layer2 und layer3. Alle Layer werden in der gegebenen Reihenfolge gezeichnet.
%		Bsp:
%			\includesvg[width=0.5\textwidth]{pic/Bilder/}{Test}{layer2, layer3}
%			Zeichnet layer2 und layer3
%	Anm: Ein alternativ kann auch \svg verwendet werden
\usepackage{legend}
\usepackage[T1]{fontenc}
\usepackage{caption}
\usepackage{siunitx}
\usepackage[utf8]{inputenc}
\usepackage[T1]{fontenc}
\usepackage{lmodern, amssymb, amsfonts}
\usepackage{mathtools}
\usepackage{siunitx} 
\usepackage{float}
\usepackage{tikz}
\usepackage{xcolor}	
\usepackage{multicol}

%%
%%%%%%%%%%

%%%%%%%%%%%%%%
%% Einstellungen für das Dokument
\iTitle[Kurz]{Developement of a Ku-Band Diplexer}
%\iSubtitle[K. U.Titel]{Untertitel}
\iAuthor[D.Autor]{Hassab Youcef}
\iDate{\today}
\iBachelor
%\iSeminar % == \iType{Seminarvortrag}
%\iType{Seminarvortrag} % Oder \iBachelor etc, siehe ihf-Template für Abschlussarbeiten

%% Beispiel für Konferenzen
%\iAuthor[D.Autor, D.Düsentrieb]{Der Author$^1$, Daniel Düsentrieb$^2$}
%\iInstitute{$^1$Technische Universität Hamburg\\
%$^2$Technische Universität Entenhausen}
%\iType{EuMW2012}
%\iPlace{Amsterdam, The Netherlands}
%\iSession{EuMIC08}

%% Mauelles Einstellen der Logos auf der Titelseite (z.B. für Konferenzen)
%\titlegraphic{\includegraphics[width=0.2\paperwidth]{./beamerthemeHarburg_tuhh_logo}
%\hspace{0.5cm}
%\includegraphics[width=0.2\paperwidth]{./ihf_logo}}


%%%%%%%%%%%%%%
%% Einstellungen für Listen:
\setbeamertemplate{itemize item}{$\bullet$} 
\setbeamertemplate{itemize subitem}{$\circ$}
\setbeamertemplate{itemize subsubitem}{$\blacktriangleright$}

%%%%%%%%%%%%%%
%% Document
\begin{document}


%%%%%%%%%%%%%%%%%%%%%%%%%%%%%%%%%%%%%%%%%%%%%%%%%%%%%%%%%%%%%%%%%%%%%%%%%%%%%%%%
% Titlepage
\titleframe
% Titlepage
%%%%%%%%%%%%%%%%%%%%%%%%%%%%%%%%%%%%%%%%%%%%%%%%%%%%%%%%%%%%%%%%%%%%%%%%%%%%%%%%

%%%%%%%%%%%%%%%%%%%%%%%%%%%%%%%%%%%%%%%%%%%%%%%%%%%%%%%%%%%%%%%%%%%%%%%%%%%%%%%%
% Table of Contents
\contentframe
% Table of Contents
%%%%%%%%%%%%%%%%%%%%%%%%%%%%%%%%%%%%%%%%%%%%%%%%%%%%%%%%%%%%%%%%%%%%%%%%%%%%%%%%

\section{Introduction}
\contentframe

\subsection{System Overiew}

\begin{frame}{Lead}
	\begin{itemize}
		\item Satellite communication
		\begin{itemize}
			\item major technological advancement
			\item vital role in communication systems
			\item many applications limited on space and mass
		\end{itemize}
		\item Filters and Diplexers
		\begin{itemize}
			\item key components in communication systems
			\item interessting signal processing propreties
		\end{itemize}
		\item Growing need for compact and efficient filters.
	\end{itemize}
\end{frame}

\begin{frame}{System Schematic}
	\begin{figure}[H]
		\svg{pic/}{system2}{}
		\caption{Simplified system schematic.}
	\end{figure}
\end{frame}	

\begin{frame}{Requirements}
\begin{itemize}
	
\item Based on dual-mode microstrip resonators
\item Integrable in stack-up
\item Compact
\item Performance:
\begin{itemize}
	\item return loss of at least 15 dB in 500 MHz bandwidths around 13.5 GHz and 17.5 GHz + minimal insertion losses.
\end{itemize}
\end{itemize}
\end{frame}	
	
\section{Basic Theory}
\contentframe

\subsection{Electrical Coupling}

\begin{frame}{Electrical Coupling}
Electrical coupling of two LC tuned circuits results in two resonance frequencies: 

\begin{equation}
f_{1} = \frac{1}{2\pi\sqrt{L(C-C_{m})}} %\quad\si{\hertz}
\end{equation}
and 
\begin{equation}
f_{2} = \frac{1}{2\pi\sqrt{L(C+C_{m})}}. %\quad\si{\hertz}
\end{equation}
    \begin{figure}
	\svg{pic/}{LC_parallel_simple}{}
	\caption{Electrically coupled resonator circuits.}
\end{figure} 
\end{frame}		

\begin{frame}{Dual-Mode Resonators}
\begin{itemize}
\item 
Support two electromagnetic modes that possess the same resonance frequency (degenerate modes)
\item Degenerate modes can be electrically coupled using a conductor perturbation
\end{itemize}
\begin{figure}[h]
	\svg{pic/}{dual_resonators}{}
	\caption{Typical microstrip dual-mode resonators.}
\end{figure} 
\end{frame}

\section{Dual-Mode Filters}
\contentframe
\begin{frame}{Layout}
	\begin{figure} 
		\centering
		\svg{pic/}{meander_loop_resonator}{}
		\caption{Layout of the dual-mode band-pass filter.}
	\end{figure}
\end{frame}

\begin{frame}{Simulation Results}
\begin{figure}
	\centering
	\svg{pic/}{2_0_filters_simulation}{}
	\caption{Simulation results \legend{pcolor2}~$S_{21}$ and \legend{pcolor1}~$S_{11}$ for the filters with \\
		(a) $13.5$ GHz central frequency and
		(b) $17.5$ GHz central frequency.}
\end{figure}
\end{frame}

\begin{frame}{sensitivities}
	\begin{multicols}{2}
		\begin{equation}
		\frac{\partial S_{11}}{\partial s} \approx \SI{0.8}{\dB\per\micro\metre}
		\end{equation}
		
		\begin{equation}
		\frac{\partial S_{11}}{\partial p} \approx \SI{0.4}{\dB\per\micro\metre}.
		\end{equation}
	\end{multicols}
\begin{figure}[H]
	\svg{pic/}{sensitivities_s_p}{}
	\caption{Effects of the variation in $s$ and $p$ on the return loss.}
\end{figure} 
\end{frame}

\begin{frame}{manufactured filters}
\begin{figure} [H]
	\svg{pic/}{produced_filters}{}
	\caption{Pictures of the produced filters.}
\end{figure}
\end{frame}

\begin{frame}{manufactured filters}
     \begin{figure}[H]
	\centering
	\svg{pic/}{13_5_S11_comparison}{}
	\caption{Comparison of the \legend{pcolor1}~simulated and \legend{pcolor2}~measured $S_{11}$ parameter of a manufactured 13.5 GHz filter.}
\end{figure} 
\end{frame}

\begin{frame}{manufactured filters}
	\begin{figure}[H]
		\centering
		\svg{pic/}{unsymmetries}{}
		\caption{Asymmetrical gaps of the manufactured filter.}
	\end{figure} 
The measured variations in the gaps are $\Delta s_{1}=14\mu m$  µm and $\Delta s_{2}=6$ m. 
The average variation is given by $\Delta s= 10$ m.

\end{frame}

\section{Compact Diplexer}
\contentframe

\section{Conclusion}
\contentframe

\end{document}

